\chapter{SQL}

SQL: Structured Query Languang

SQL语言的特点:
\begin{enumerate}
    \item 语言简洁,易学易用。
    \item 面向集合的操作方式,一次一集合。
    \item 高度非过程化。\textit{用户只需提出“做什么” ,无须告诉“怎么做”}
    \item 一体化。单一的结构——关系,带来了数据操作符的统一。
    \item 两种使用方式,统一的语法结构。
    \begin{enumerate}
        \item 既是自含式(用户使用)的
        \item 又是嵌入式的(程序员使用)
    \end{enumerate}
\end{enumerate}

\begin{table}[H]
  \centering
  \begin{tabular}{|c|c|}
    \hline
    \textbf{SQL功能} & \textbf{操作符} \\
    \hline
    数据定义 & \verb|create|, \verb|alter|, \verb|drop| \\
    \hline
    数据查询 & \verb|select| \\
    \hline
    数据修改 & \verb|insert|, \verb|update|, \verb|delete| \\
    \hline
    数据控制 & \verb|grant|, \verb|revoke| \\
    \hline
  \end{tabular}
  \caption{SQL主要操作符}
\end{table}

text2SQL: 建立自然语言与结构化数据之间的关系。

\section{数据模式定义}

\begin{figure}[H]
  \centering
  \begin{minipage}[t]{0.45\textwidth}
    \centering
    \includegraphics[width=\linewidth]{./figure/标准SQL.pdf}
    \caption{标准SQL中的数据定义对象}
  \end{minipage}
  \hfill
  \begin{minipage}[t]{0.5\textwidth}
    \centering
    \includegraphics[width=\textwidth]{./figure/实际.pdf}
    \caption{实际数据库(SQL Server)中的定义对象}
  \end{minipage}
\end{figure}

SQL Server: 模式把对象和用户分离开来.

对象命名: \verb|<数据库>.<模式>.<表>|.

创建模式:
\begin{lstlisting}[language=SQL]
create schema <模式名>
create schema University.Library
\end{lstlisting}

数据库定义: SQL Server
\begin{lstlisting}[language=SQL]
create database <数据库名>
  [on [primary] <文件描述> <文件组> ...]
  [log on <文件描述> <文件组> ...]
\end{lstlisting}

最简单的创建数据库的命令: \verb|create databse University|.

\verb|use| 命令指定当前要使用的数据库: \verb|use University|.

\begin{lstlisting}[language=SQL]
create database demoDB1
on primary
( name = demo_dat1,
  filename = 'D:\SQL_Practice\demodata1.mdf',
  size = 10,
  maxsize = 50)
log on
(
  name = demo_log1,
  filename = 'D:\SQL_Practice\demodata1.ldf',
  size = 5,
  filegrowth = 5
)
\end{lstlisting}

数据库定义: MySQL
\begin{lstlisting}[language=SQL]
create database <数据库名>
  [ default character set utf8
    default collate utf8_Chinese_ci ]
\end{lstlisting}

\verb|create database| 等于 \verb|create schema|.

MySQL表空间:

\begin{figure}[H]
    \centering
    \includegraphics[width=.45\textwidth]{./figure/MySQL表空间.pdf}
    \caption{MySQL表空间}
\end{figure}

\begin{lstlisting}[language=SQL]
create tablespace myTs 'ts1.ibd' engine = innodb
create tablespace myTs add datafile
  'F:\\test_mysql_tablespace\\first.ibd'
create table myTb (...) tablespace myTs
\end{lstlisting}

创建基本表的语法命令:
\begin{lstlisting}[language=SQL]
create table <表名> (
  <列名> <数据类型> [default <缺省值>] [not null] [unique]
  [, <列名> <数据类型> [default <缺省值>] [not null] [unique]]
  ...
  [, primary key (<列名> [, <列名>] ...)]
  [, foreign key (<列名> [, <列名>] ...)
    references <表名> (<列名> [, <列名>] ...)]
  [, check(<条件>)]
)
\end{lstlisting}

下面是创建表的一些例子:
\begin{lstlisting}[language=SQL]
create table student
( sno char(8),
  sname char(8) not null default '佚名',
  age tinyint,
  sex char(1),
  primary key (sno),
  check (sex = 'M' or sex='F')
)
\end{lstlisting}

\begin{lstlisting}[language=SQL]
create table course
( cno char(8) primary key,
  cname char(8) not null unique,
  pcno char(8) foreign key references C(cno),
  credit tinyint
)
\end{lstlisting}

\begin{lstlisting}[language=SQL]
create table SC
( sno char(8) foreign key references S(sno),
  cno char(8) foreign key references C(cno),
  grade tinyint,
  primary key (sno, cno),
  check((grade is null) or grade between 0 and 100)
)
\end{lstlisting}

修改基本表: 更改、添加、除去列和约束.
\begin{lstlisting}[language=SQL]
alter table <表名>
  [add column <子句>]
  [add constraint <子句>]
  [drop <子句>]
  [alter column <子句>]
\end{lstlisting}

\begin{lstlisting}[language=SQL]
-- 在student表age列之后加入addr
alter table student add column addr CHAR(30) after age;
-- 把addr列重命名为address
alter table student change addr address CHAR(50) not null;
-- 试修改teacher表中的salary列的数据类型为bigint
alter table teacher modify salary bigint;
-- 重命名一个表中的列名从sal到salary
alter table rename sal to salary
\end{lstlisting}

删除基本表:
\begin{lstlisting}[language=SQL]
drop table <表名>;
\end{lstlisting}
删除表定义及该表的所有数据、索引、触发器、约束和权限规范.

drop table不能删除被foreign key约束所引用的表, 必须先除去foreign key约束或引用表.

任何引用已删除表的视图或存储过程必须通过drop view或drop procedure语句显式除去.

标准SQL中的信息视图:
\begin{lstlisting}[language=SQL]
INFORMATION_SCHEMA.SCHEMATA
INFORMATION_SCHEMA.TABLES
INFORMATION_SCHEMA.COLUMNS
INFORMATION_SCHEMA.CHECK_CONSTRAINTS
INFORMATION_SCHEMA.VIEWS
INFORMATION_SCHEMA.DOMAINS
\end{lstlisting}

MySQL中的信息视图查询:
\begin{lstlisting}[language=SQL]
select shcema_name from information_schema.schemata;
select table_name from information_schema.tables;
select column_name from information_schema.columns where table_name = 'student';
\end{lstlisting}

\begin{table}[H]
\centering
\label{tab:sysobjects}
\begin{tabular}{|l|l|l|}
\hline
\multicolumn{3}{|c|}{sysobjects} \\ \hline
\textbf{列名} & \textbf{数据类型} & \textbf{描述} \\ \hline
name & sysname & 对象名 \\ \hline
Id & int & 对象标识号 \\ \hline
xtype & char(2) & 对象类型 \\ \hline
uid & smallint & 所有者对象的用户ID \\ \hline
crdate & datetime & 对象的创建日期 \\ \hline
schema\_ver & int & 版本号,该版本号在每次表的架构更改时都增加 \\ \hline
\end{tabular}
\caption{表定义相关的字典表: SQL Server}
\end{table}

\begin{table}[H]
\centering
\label{tab:syscolumns}
\begin{tabular}{|l|l|l|}
\hline
\multicolumn{3}{|c|}{syscolumns} \\ \hline
\textbf{列名} & \textbf{数据类型} & \textbf{描述} \\ \hline
name & sysname & 列名或过程参数的名称 \\ \hline
id & int & 该列所属的表对象ID \\ \hline
 xtype & tinyint & systypes 中的物理存储类型 \\ \hline
xusertype & smallint & 扩展的用户定义数据类型ID \\ \hline
length & smallint & systypes 中的最大物理存储长度 \\ \hline
offset & smallint & 该列所在行的偏移量;如果为负,表示可变长度行 \\ \hline
type & tinyint & systypes 中的物理存储类型 \\ \hline
usertype & smallint & systypes 中的用户定义数据类型ID \\ \hline
isnullable & int & 表示该列是否允许空值 \\ \hline
\end{tabular}
\caption{表定义相关的字典表: SQL Server}
\end{table}

SQL中, 任何时候都可以执行一个数据定义语句, 随时修改数据库结构.

\subsection{数据类型}

\begin{table}[H]
\centering
\label{tab:mysql_int_types}
\begin{tabular}{|l|l|l|l|}
\hline
\textbf{数据类型} & \textbf{范围} & \textbf{unsigned范围} & \textbf{存储字节数} \\ \hline
tinyint & $-2^7 \sim 2^7 - 1$ & $0 \sim 2^8 - 1$ & 1字节 \\ \hline
smallint & $-2^{15} \sim 2^{15} - 1$ & $0 \sim 2^{16} - 1$ & 2字节 \\ \hline
mediumint & $-2^{23} \sim 2^{23} - 1$ & $0 \sim 2^{24} - 1$ & 3字节 \\ \hline
int & $-2^{31} \sim 2^{31} - 1$ & $0 \sim 2^{32} - 1$ & 4字节 \\ \hline
bigint & $-2^{63} \sim 2^{63} - 1$ & $0 \sim 2^{64} - 1$ & 8字节 \\ \hline
\end{tabular}
\caption{MySQL整数数据类型及其范围和存储字节数}
\end{table}

\begin{lstlisting}[language=SQL]
create table test_int (
  a(6) tinyint zerofill,
  b(6) tinyint unsigned );
insert into test_int values (1, 111);
select a, b from test_int;
-- a 000001 b 111
select a - b from test_int;
-- ERROR 1690 (22003): BIGINT UNSIGNED value is out of range
\end{lstlisting}

宽松模式: \verb|set sql_mode = 'ANSI'|. 对于违反数据约束的有一些默认操作.

严格模式: \verb|set sql_mode = 'traditional'|. 直接报错.

