\section{24春季期末部分考试题目}

\subsection{第一题}

\begin{enumerate}
    \item 三级模式.
    \begin{enumerate}
        \item 虚视图: 外模式;
        \item 实视图: 内模式(因为是真正存储起来)
        \item 索引是什么模式: 内模式
    \end{enumerate}
    \item 外码的三种定义.
    \item 任意一个包含三元多对多联系的ER图, 最少的关系模式数量
    \item $R(A,B,C)$有10个元组, 非null, 求 $\Pi_{AB}(R) \bowtie \Pi_{AC}(R)$元组数目范围.
    \item $A\to B$成立, 关系代数; $A\to \to B$成立, 关系代数.
    \item 避免出现转换死锁, 最高隔离级别.
    \item 对于关系模式$R(ABCDEF)$, $F=\{A\to B,C\to D, E\to F\}$, 无损BCNF分解有\underline{\quad}种分解结果.
    \item $X\to Y, WY\to Z$, 则$WX\to Z$, 这条规则叫\underline{\quad}, 写出Armstrong公理体系的增广律: \underline{\quad}.
    \item 属性全是主属性, 最高一定可达范式; 全码关系模式, 最高一定可达范式.
    \item 避免死锁转换, 最高隔离级别; 避免发生丢失修改, 最低隔离级别.
    \item 两段锁协议. 调度可串协议; 顺序和事务\underline{\quad}顺序一致, 最低割裂级别.
    \item ARIES: \underline{\quad \quad \quad}.
    \item RAID的区别, 最快的是哪个
    \item 3种DB连接: 非等值连接, 连表时较快
    \item 元组数目范围
\end{enumerate}

\subsection{第二题}

\begin{enumerate}
    \item 元组关系验算表达式$\to$域关系验算表达式
    \item 数据库安全性相关属于
    \item IS IU相容
    \item 以下哪些可以提升事务吞吐率: 1. 提升隔离性级别; 2.提升封锁度; 3.提升checkpoint生成频率; 4.不同事务相反顺序访问数据项; 5.大事务分解成多个小事务; 6.使用索引
    \item $R(A)=\{1,2\}; S(B)=\{3,4,5,6\}$. create trigger 名字 before insert on A.
\end{enumerate}

\subsection{三、关系代数}

\begin{enumerate}
    \item 除法
    \item $R(A, B), S(B, C)$左外连接, 关系代数
    \item $R(A)$, 用关系代数求$A_{\min}$
    \item $C(sno, cno)$. 选修课程集合被S0所选集合包含
\end{enumerate}

\subsection{四、SQL}

\verb|stock (stock_no, day_time, price)|.
\begin{enumerate}
    \item 涨跌
    \item 停牌天数
    \item 哪一天有最多支股票涨停
\end{enumerate}
